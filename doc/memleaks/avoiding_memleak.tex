\documentclass[12pt]{article}


\begin{document}

\title{Object-oriented approach in Fortran 95: avoiding memory leaks with derived data types}
\author{Anna Kelbert}
\maketitle


\section*{Abstract}

This document describes our approach to the typical memory leaks and
other problems commonly encountered in Fortran 95 when dealing with
derived data types that contain allocatable/pointer components. When
the Technical Report ISO/IEC 15581 (the "allocatable array
extension" to F95) is adopted, these solutions will hopefully become
obsolete. Currently, most state of the art compilers (g95, ifort,
pgf95) do not support these extensions, making a temporary solution
such as that described here crucial. We strongly recommend that
these practices are carefully followed for all future development of
this code, until they are proven to be unnecessary. In the
following, when we refer to a data type, we mean a data type with a
pointer component. Creating such a data type involves a memory
allocation procedure.

\section*{Rule 1} \textbf{Interface assignment (a = b) has to be
overloaded.}\\

\emph{Reason:} When not overloaded, the pointers inside \textit{a}
point to the same memory location as those in \textit{b}. Thus,
vectors are not copied, and deallocating \textit{b} later in the
code will create an unusable \textit{a} structure. Very dangerous.

\section*{Rule 2} \textbf{Functions that output a data type have to
\begin{enumerate}
    \item allocate the output, and
    \item mark the output as temporary.
\end{enumerate}
Use subroutines over functions where ever efficiency is an issue.}\\

\emph{Reason:} On input to the function, the output data type is
completely undefined. Thus, checking it's
\verb"allocated"/\verb"associated" status is meaningless. It
therefore has to be allocated inside the function. Once allocated,
the result of the function is stored in a temporary variable, call
it RHS. Since the equals (=) sign is overloaded, it is then passed
as input to the user-defined copy subroutine that overloads the
assignment. The LHS is computed, but the RHS is still stored in
memory. Moreover, once we exit the copy subroutine, it can no longer
be accessed or deallocated. Most compilers (certainly g95 and ifort
at the time of writing) do not deallocate these temporary variables.
This creates very real memory leaks, i.e. every time such a function
is called, a copy of the data type is left behind in system memory.
This can very easily result in unusable code.\\

\emph{Example:} \verb"a = add(b,c)" OR \verb"a = b + c" OR \verb"a = zero(b)"\\

\emph{Solution:} One possibility is staying away from functions (and
hence, from overloaded operators!). We adopt a different approach.
Every data type contains a logical variable \verb"temporary",
\verb".false." by default. It is only \verb".true." for a function
output. The \verb"copy" subroutine deallocates it's input if and
only if it is "temporary". This can be done be tricking the
compiler: the \verb"deall" subroutine does not have an intent for
it's input argument (note that \verb"intent(inout)" would make the
compiler bomb due to the \verb"intent(in)" specification of the
variable in the \verb"copy" subroutine). This works and gets rid of
the memory leaks; it also allows us to safely use functions. When
this trick becomes unnecessary as the compilers develop, it would be
easy to get rid of
it in the code.\\

\emph{Restrictions:}

With this approach, usage of the functions prohibits the following.

\emph{Example:} \verb"write(0,*) add(b,c)"

\emph{Reason:} The temporary variable is created and not
deallocated, causing a memory leak.\\

\emph{Example:} \verb"a = b + c*d"

\emph{Reason:} The temporary variable c*d is created and not
deallocated, causing a memory leak. So, no concatenation of
operators! This would be allowed if the above approach is extended
to delete temporary input variables in all functions. However, that
requires a bit of extra
coding. Besides, for efficiency, this usage is discouraged.\\

\emph{Example:} usage of functions in the lower level code

\emph{Reason:} All the temporary variables that are created in the
process may affect the efficiency of the code. Thus, good
programming disallows usage of any functions in a subroutine that is
going to be called many times in the code. Only use functions and/or
operators in the upper level code, where readability is more
important than efficiency. Use subroutines (such as \verb"linComb")
otherwise.


\section*{Rule 3} \textbf{No data type intent(out) dummy arguments
in subroutines.}\\

\emph{Reason:} For a mysterious reason that I do not understand, the
compilers also create a temporary variables for such a data type
(just like for function outputs), that cannot be deallocated after
exiting the subroutine, creating a steady memory leak.\\

\emph{Solution:} Use \verb"intent(inout)" instead.


\section*{Rule 4} \textbf{Basic subroutines require that
intent(inout) dummy arguments are allocated and of the correct
size.}\\

\emph{Reason:} Variables are passed by reference in Fortran. Thus,
in a call such as \verb"linComb(a1,v1,a2,v2,v1)" that overwrites its
own input variable, any modification (or deallocation) of the output
\verb"v1" inside the subroutine also modifies (or deallocates) the
input \verb"v1". Mysterious errors result. Not a memory leak issue,
but may be worse. Impossible for
the compiler to track.\\

\emph{Solution:} Such subroutines may not reuse their output. So, no
allocation/deallocation or recursion in subroutines that may
potentially overwrite their inputs. This is our solution for this
code.\\

\emph{An alternative solution:} Require the output to NOT exist at
the entrance to the subroutine. Drawbacks of this approach include
inefficiency, since no outputs may overwrite the inputs. Hence, more
temporary variables will be floating around, resulting in multiple
additional allocation/deallocation calls. However, programming is
safer.


\section*{Rule 5} \textbf{Always deallocate temporary data types
before exiting a subroutine.}\\

\emph{Reason:} To avoid multiple memory leaks. A data type with
pointer components is not automatically deallocated on exit from a
subroutine.


\section*{Rule 6} \textbf{Only allocate a pointer if it's not already
associated.}\\

\emph{Reason:} Statements such as \verb"allocate(vector,STAT=istat)"
do not save you from allocating the memory again. If the vector was
already allocated, this statement results in a memory leak. The only
fail proof check is \verb"associated(vector)".

\end{document}
